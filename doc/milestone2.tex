% Document Type: LaTeX
\documentclass[11pt]{article}

\usepackage[parfill]{parskip}
\usepackage{listings}
\usepackage[style=ieee]{biblatex}
\addbibresource{references.bib}

%%% Uncomment any of the macro packages below if you need them.

%% Math symbols.  Probably these are needed if you are doing logic
\usepackage{amsmath,amssymb,latexsym,stmaryrd,amsthm}

%% Best drawing package ever
%\usepackage{tikz}
%\usetikzlibrary{automata,arrows}

%% For including pictures in ps, eps, pdf or jpeg etc.
\usepackage{graphicx}

%% Use at your own peril
%\usepackage{pstricks}

%% Good for typesetting proof trees
%\usepackage{proof}
\usepackage{bussproofs}
%% Don't use colours unless you have to.
%\usepackage{color}

%% This is for checking the format.  It will generate pages of drivel that
%% vaguely resembles a comic-book parody of Latin.
\usepackage{lipsum}

\newtheorem{defin}{Definition}

%% If you create macros (and you should) uncomment this.
%\include{macros}
\begin{document}
% \bibliographystyle{alpha} % 

\title{Milestone 1: Go Parser}
\author{Xavier Denis}
\date{\today}
\maketitle

\section {Design Decisions}
While Haskell provides a natural interface for many recursive tasks, building a typed AST and the requirements to return the symbol table posed a few challenges. 

In the Haskell community the AST Typing problem has no definitive solution. Haskell itself uses an explicitly typed AST which gets built during typechecking. Other approaches include building an infinite type using Fixed points of the types and exploiting the Cofree-Comonad (whatever that means). The downside of these approaches is that it makes pattern matching much more difficult.  Instead, I opted to use an intermediate approach, that is I parameterized all types containing `Expression' over `a'. By adding `a' to all the constructors of `Expression' I can attach any information I wish to expressions.

I chose to model the actual typechecker through another typeclass called `Checkable'. This simplifies the code slightly, since all checker methods share a name. It fits nice into the semantics of typeclasses which are used to express a common behaviour shared by types.
\begin{figure}
\begin{lstlisting}
type Check = ExceptT TypeError (State (String, Counters, Env))
type Ann = Type
class Checkable a where
  check :: a -> (Check (a Ann))
\end{lstlisting}
\end{figure}

The definition given above, which is used in the code causes a problem. The type `a' has kind $* \rightarrow *$, however, Haskell does not support polymorphic instance declarations for higher kinded types by default. In my previous typeclasses (Pretty, Weeder), I exploited these polymorphic instances to simplify dealing with `Maybe a' and `[a]'. This meant that the code would be slightly more verbose than was strictly required. 

Returning the symbol table on scope exit causes a few problems of it's own, most notably it unreasonable to not use the `State' monad to carry the log through the check. However, since the `Either' monad was already being used to represent abortable computation, monad transformers were required to combine both monads together. While monad transformers are great tools, they can come with a performance penalty of up to $300\%$.

\section {Scoping Rules}
There are 3 major scoping cases:
\begin {enumerate}
  \item Functions: When functions are created, we open a scope to push all the parameters into. Since function bodies are represented by a block statement and would normally open another scope, we manually read out the statements in the function to avoid parameter shadowing.
  \item If/Switch/For: These statements can have statements in their condition clauses, so we open a scope to process these statements.
  \item Block: Block statements are used to represent all blocks in curly braces, including If statements or For loops. These open a scope when they process and pop it when the close.
\end {enumerate}

\section {Type Rules}
There are a \textit{many} type rules used. Briefly they can fall into 3 major sections:
\begin {enumerate}
  \item Type: These checks focused on properly adding types to the symbol table. Additionally, they deal with all the scoping issues that happen when nested types share the same name. Relevant tests: 
    \begin{enumerate}
      \item INVALID/alias.go
      \item INVALID/tricky\_alias.go
      \item INVALID/invalid\_if.go
      \item VALID/tricky\_alias.go
      \item VALID/alias.go
      \item VALID/var\_scopes.go
    \end{enumerate}
  \item Expression: These checks focus on checking all expressions according the go and golite specs. They also make sure to assign the correct type to the expression so that future operations succeed. Relevant tests:
    \begin{enumerate}
      \item INVALID/invalid\_comp.go
      \item INVALID/assign.go
      \item VALID/valid\_slice.go \textit{append is treated as a function invocation (expression)}
      \item VALID/var\_scopes.go
    \end{enumerate}
  \item Statement: These checks are mainly focused on correct recursion. Certain statements such as for/if/switch also require secondary checks. Relevant tests:
    \begin{enumerate}
      \item INVALID/invalid\_dec.go
      \item INVALID/invalid\_if.go
      \item VALID/for.go
      \item VALID/valid\_slice.go
      \item VALID/valid\_switch.go
    \end{enumerate}
\end {enumerate}
\section {Work Split}

I, Xavier Denis, did this entire milestone.

\end{document}